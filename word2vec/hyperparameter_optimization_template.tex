\documentclass[10pt,letterpaper]{article}

% Language and encoding:
\usepackage[english]{babel}
\usepackage[utf8]{inputenc}

% Sets page size and margins:
\usepackage[a4paper,top=3cm,bottom=2cm,left=3cm,right=3cm,marginparwidth=1.75cm]{geometry}

% Useful packages:
\usepackage{amsmath}
\usepackage{graphicx}
\usepackage{authblk}
\usepackage{pgfplots}
\DeclareUnicodeCharacter{2212}{−}
\usepgfplotslibrary{groupplots,dateplot}
\usetikzlibrary{patterns,shapes.arrows}
\pgfplotsset{compat=newest}
\usepackage[hidelinks]{hyperref}

% Title
\title{
		%\vspace{-1in} 	
		\usefont{OT1}{bch}{b}{n}
		\normalfont \normalsize \textsc{Discovering latent knowledge in medical papers on Acute Myeloid Leukemia\\ FAPESP project \{doubleblind\}} \\ [10pt]
		\huge \VAR{title} Hyperparameter Optimization \\
}

\author[1]{doubleblind}

\affil[1]{\small{doubleblind}}

\begin{document}
\maketitle
\selectlanguage{english}

\section{Settings}
\begin{table}[ht]
    \centering
    \caption{Fixed hyperparameters values.}
    \label{tab:fixed_hyperparameters}
    \begin{tabular}[t]{lcc}
        \hline
        Hyperparameter & Value\\
        \hline
        hierarchical softmax & 0 \\
        iter & 15 \\
        min count & 5 \\
        sg & 1 \\
        sorted vocab & True \\
	window & 5 \\
        \hline
    \end{tabular}
\end{table}

\VAR{model_names_filepaths_df}

\VAR{hyperparameter_optimization_df}

The "Default" model was trained using the original hyperparameters defined in implementation of the model by Gensim package.\footnote{Word2Vec: \url{https://radimrehurek.com/gensim/models/word2vec.html\#gensim.models.word2vec.Word2Vec} or FastText: \url{https://radimrehurek.com/gensim_3.8.3/models/fasttext.html\#gensim.models.fasttext.FastText}}

\section{Performance}
\begin{table}[ht]
\centering
\caption{Amount of analogies per type.}
\label{tab:analogies_amount}
\begin{tabular}{llr}
 & Analogies & Amount \\
0 & All & \VAR{number_analogies_all} \\
1 & AML & \VAR{number_analogies_AML} \\
2 & Grammar & \VAR{number_analogies_general} \\
2 & Biomedical & \VAR{number_analogies_biomedical} \\
\end{tabular}
\end{table}

\BLOCK{for table in performance_dfs}
	\VAR{table}
\BLOCK{endfor}

\BLOCK{for plot, caption in performance_plots_and_captions}
	\begin{figure}[ht]
	\centering
	\VAR{plot}
	\caption{\VAR{caption}}
	\end{figure}
\BLOCK{endfor}

\end{document}
